\documentclass{article}
\usepackage[utf8]{inputenc}

\title{PS10_Patel}
\author{shreyapatel }
\date{April 2018}

\begin{document}


\maketitle{#7}

\begin{tabular}{lrr}
\toprule
  & f1 & gmean\\
  
\midrule
tree   & 0.8976425 & 2094.198\\
logit  & 0.8633508 & 0.000\\
nnet   & 0.9069514 & 2126.452\\
knn    & 0.8977964 & 2106.887\\
svm    & 0.8939016 & 2086.292\\
nbayes & 0.8825952 & 2043.110\\

\bottomrule
\end{tabular}
% 

\\

\maketitle{#8}

\begin{tabular}{lll}
\toprule
value & measure & model\\
\midrule
21        & minsplit  & tree\\
12        & minbucket & tree\\
0.003359  & cp        & tree\\
24        & k         & knn\\
2.354535  & lambda    & logit\\
\addlinespace
0.472230  & alpha     & logit\\
radial    & kernel    & svm\\
47.001022 & cost      & svm\\
0.498152  & gamma     & svm\\
10        & size      & nnet\\
\addlinespace
0.457640  & decay     & nnet\\
1000      & maxit     & nnet\\
\bottomrule
\end{tabular}

\\
\maketitle{#9}\\


As a table in your .tex file, report the optimal values of the tuning parameters for each of the algorithms. How does each algorithm’s out-of-sample performance compare with each of the other algorithms?\\

Most of them are similar, especially when looking at F values. They are more spread out in G values. 

\end{document}

